\documentclass[journal,12pt,twocolumn]{IEEEtran}
%
\usepackage{setspace}
\usepackage{gensymb}
\usepackage{xcolor}
\usepackage{caption}
%\usepackage{subcaption}
%\doublespacing
\singlespacing
\usepackage{enumitem}
%\usepackage{multicol}
%\usepackage{graphicx}
%\usepackage{amssymb}
%\usepackage{relsize}
\usepackage[cmex10]{amsmath}
\usepackage{mathtools}
%\usepackage{amsthm}
%\interdisplaylinepenalty=2500
%\savesymbol{iint}
%\usepackage{txfonts}
%\restoresymbol{TXF}{iint}
%\usepackage{wasysym}
\usepackage{amsthm}
\usepackage{mathrsfs}
\usepackage{txfonts}
\usepackage{stfloats}
\usepackage{cite}
\usepackage{cases}
\usepackage{subfig}
%\usepackage{xtab}
\usepackage{longtable}
\usepackage{multirow}
%\usepackage{algorithm}
%\usepackage{algpseudocode}
\usepackage{enumitem}
\usepackage{mathtools}
\usepackage{iithtlc}
%\usepackage[framemethod=tikz]{mdframed}
\usepackage{listings}
    \usepackage[latin1]{inputenc}                                 %%
    \usepackage{color}                                            %%
    \usepackage{array}                                            %%
    \usepackage{longtable}                                        %%
    \usepackage{calc}                                             %%
    \usepackage{multirow}                                         %%
    \usepackage{hhline}                                           %%
    \usepackage{ifthen}                                           %%
  %optionally (for landscape tables embedded in another document): %%
    \usepackage{lscape}     

%\usepackage{stmaryrd}


%\usepackage{wasysym}
%\newcounter{MYtempeqncnt}
\DeclareMathOperator*{\Res}{Res}
%\renewcommand{\baselinestretch}{2}
\renewcommand\thesection{\arabic{section}}
\renewcommand\thesubsection{\thesection.\arabic{subsection}}
\renewcommand\thesubsubsection{\thesubsection.\arabic{subsubsection}}

\renewcommand\thesectiondis{\arabic{section}}
\renewcommand\thesubsectiondis{\thesectiondis.\arabic{subsection}}
\renewcommand\thesubsubsectiondis{\thesubsectiondis.\arabic{subsubsection}}

% correct bad hyphenation here
\hyphenation{op-tical net-works semi-conduc-tor}

\def\inputGnumericTable{}  

\lstset{
%language=python,
frame=single, 
breaklines=true,
columns=fullflexible
}
\newcommand\bigzero{\makebox(0,0){\text{\huge0}}}
%\lstset{
	%%basicstyle=\small\ttfamily\bfseries,
	%%numberstyle=\small\ttfamily,
	%language=Octave,
	%backgroundcolor=\color{white},
	%%frame=single,
	%%keywordstyle=\bfseries,
	%%breaklines=true,
	%%showstringspaces=false,
	%%xleftmargin=-10mm,
	%%aboveskip=-1mm,
	%%belowskip=0mm
%}

%\surroundwithmdframed[width=\columnwidth]{lstlisting}


\begin{document}
%

\theoremstyle{definition}
\newtheorem{theorem}{Theorem}[section]
\newtheorem{problem}{Problem}
\newtheorem{proposition}{Proposition}[section]
\newtheorem{lemma}{Lemma}[section]
\newtheorem{corollary}[theorem]{Corollary}
\newtheorem{example}{Example}[section]
\newtheorem{definition}{Definition}[section]
%\newtheorem{algorithm}{Algorithm}[section]
%\newtheorem{cor}{Corollary}
\newcommand{\BEQA}{\begin{eqnarray}}
\newcommand{\EEQA}{\end{eqnarray}}
\newcommand{\define}{\stackrel{\triangle}{=}}

\bibliographystyle{IEEEtran}
%\bibliographystyle{ieeetr}

\providecommand{\nCr}[2]{\,^{#1}C_{#2}} % nCr
\providecommand{\nPr}[2]{\,^{#1}P_{#2}} % nPr
\providecommand{\mbf}{\mathbf}
\providecommand{\pr}[1]{\ensuremath{\Pr\left(#1\right)}}
\providecommand{\qfunc}[1]{\ensuremath{Q\left(#1\right)}}
\providecommand{\sbrak}[1]{\ensuremath{{}\left[#1\right]}}
\providecommand{\lsbrak}[1]{\ensuremath{{}\left[#1\right.}}
\providecommand{\rsbrak}[1]{\ensuremath{{}\left.#1\right]}}
\providecommand{\brak}[1]{\ensuremath{\left(#1\right)}}
\providecommand{\lbrak}[1]{\ensuremath{\left(#1\right.}}
\providecommand{\rbrak}[1]{\ensuremath{\left.#1\right)}}
\providecommand{\cbrak}[1]{\ensuremath{\left\{#1\right\}}}
\providecommand{\lcbrak}[1]{\ensuremath{\left\{#1\right.}}
\providecommand{\rcbrak}[1]{\ensuremath{\left.#1\right\}}}
\theoremstyle{remark}
\newtheorem{rem}{Remark}
\newcommand{\sgn}{\mathop{\mathrm{sgn}}}
\providecommand{\abs}[1]{\left\vert#1\right\vert}
\providecommand{\res}[1]{\Res\displaylimits_{#1}} 
\providecommand{\norm}[1]{\lVert#1\rVert}
\providecommand{\mtx}[1]{\mathbf{#1}}
\providecommand{\mean}[1]{E\left[ #1 \right]}
\providecommand{\fourier}{\overset{\mathcal{F}}{ \rightleftharpoons}}
%\providecommand{\hilbert}{\overset{\mathcal{H}}{ \rightleftharpoons}}
\providecommand{\system}{\overset{\mathcal{H}}{ \longleftrightarrow}}
	%\newcommand{\solution}[2]{\textbf{Solution:}{#1}}
\newcommand{\solution}{\noindent \textbf{Solution: }}
\providecommand{\dec}[2]{\ensuremath{\overset{#1}{\underset{#2}{\gtrless}}}}
%\numberwithin{equation}{subsection}
\numberwithin{equation}{section}
%\numberwithin{equation}{problem}
%\numberwithin{problem}{subsection}
\numberwithin{problem}{section}
%\numberwithin{definition}{subsection}
\makeatletter
\@addtoreset{figure}{problem}
\makeatother
\makeatletter
\@addtoreset{table}{problem}
\makeatother

\let\StandardTheFigure\thefigure
\let\StandardTheTable\thetable
%\renewcommand{\thefigure}{\theproblem.\arabic{figure}}
\renewcommand{\thefigure}{\theproblem}
\renewcommand{\thetable}{\theproblem}
%\numberwithin{figure}{section}

%\numberwithin{figure}{subsection}

\def\putbox#1#2#3{\makebox[0in][l]{\makebox[#1][l]{}\raisebox{\baselineskip}[0in][0in]{\raisebox{#2}[0in][0in]{#3}}}}
     \def\rightbox#1{\makebox[0in][r]{#1}}
     \def\centbox#1{\makebox[0in]{#1}}
     \def\topbox#1{\raisebox{-\baselineskip}[0in][0in]{#1}}
     \def\midbox#1{\raisebox{-0.5\baselineskip}[0in][0in]{#1}}

\vspace{3cm}


\title{%Convex Optimization in Python
	\logo{
	KKT Conditions: Support Vector Machines
	}
}
%\title{
%	\logo{Matrix Analysis through Octave}{\begin{center}\includegraphics[scale=.24]{tlc}\end{center}}{}{HAMDSP}
%}


% paper title
% can use linebreaks \\ within to get better formatting as desired
%\title{Matrix Analysis through Octave}
%
%
% author names and IEEE memberships
% note positions of commas and nonbreaking spaces ( ~ ) LaTeX will not break
% a structure at a ~ so this keeps an author's name from being broken across
% two lines.
% use \thanks{} to gain access to the first footnote area
% a separate \thanks must be used for each paragraph as LaTeX2e's \thanks
% was not built to handle multiple paragraphs
%

\author{Y Aditya, G V S S Praneeth Varma and G V V Sharma$^{*}$% <-this % stops a space
\thanks{* The authors are with the Department
of Electrical Engineering, Indian Institute of Technology, Hyderabad
502285 India e-mail:  gadepall@iith.ac.in.}% <-this % stops a space
%\thanks{J. Doe and J. Doe are with Anonymous University.}% <-this % stops a space
%\thanks{Manuscript received April 19, 2005; revised January 11, 2007.}}
}
% note the % following the last \IEEEmembership and also \thanks - 
% these prevent an unwanted space from occurring between the last author name
% and the end of the author line. i.e., if you had this:
% 
% \author{....lastname \thanks{...} \thanks{...} }
%                     ^------------^------------^----Do not want these spaces!
%
% a space would be appended to the last name and could cause every name on that
% line to be shifted left slightly. This is one of those "LaTeX things". For
% instance, "\textbf{A} \textbf{B}" will typeset as "A B" not "AB". To get
% "AB" then you have to do: "\textbf{A}\textbf{B}"
% \thanks is no different in this regard, so shield the last } of each \thanks
% that ends a line with a % and do not let a space in before the next \thanks.
% Spaces after \IEEEmembership other than the last one are OK (and needed) as
% you are supposed to have spaces between the names. For what it is worth,
% this is a minor point as most people would not even notice if the said evil
% space somehow managed to creep in.



% The paper headers
%\markboth{Journal of \LaTeX\ Class Files,~Vol.~6, No.~1, January~2007}%
%{Shell \MakeLowercase{\textit{et al.}}: Bare Demo of IEEEtran.cls for Journals}
% The only time the second header will appear is for the odd numbered pages
% after the title page when using the twoside option.
% 
% *** Note that you probably will NOT want to include the author's ***
% *** name in the headers of peer review papers.                   ***
% You can use \ifCLASSOPTIONpeerreview for conditional compilation here if
% you desire.




% If you want to put a publisher's ID mark on the page you can do it like
% this:
%\IEEEpubid{0000--0000/00\$00.00~\copyright~2007 IEEE}
% Remember, if you use this you must call \IEEEpubidadjcol in the second
% column for its text to clear the IEEEpubid mark.



% make the title area
\maketitle

\tableofcontents

\renewcommand{\thefigure}{\theenumi}
\renewcommand{\thetable}{\theenumi}

\begin{abstract}
This manual provides a simple introduction to various concepts in optimization.
\end{abstract}



\section{Lagrangian}
\begin{enumerate}[label=\thesection.\arabic*,ref=\thesection.\theenumi]

\item
	\label{convex_code}
	Plot the circles 
%
\begin{equation}
\label{eq2_1_circ}
f\brak{\mbf{x}} = (x_1-8)^2 + (x_2-6)^2 = r^2
\end{equation}
%
 $\mbf{x}= \brak{x_1,x_2}^{T}$, for different values of $r$ along with the line 
%
\begin{equation}
\label{eq2_1_line}
g\brak{\mbf{x}} = x_1 + x_2 - 9 = 0
\end{equation} 
%
using the following program.	From the graph, find
\begin{align}
	\min_{\mbf{x}}f\brak{\mbf{x}} \quad \text{s.t.} \\
	g\brak{\mbf{x}} = x_1 + x_2 - 9 = 0
\end{align}
%

%	
\begin{lstlisting}
wget https://raw.githubusercontent.com/gadepall/EE2250/master/manual/codes/2.1.py
\end{lstlisting}

%
\begin{figure}[!ht]
\centering
\includegraphics[width=\columnwidth]{./figs/2.1.eps}
\caption{ Finding $ \displaystyle \min_{\mbf{x}}f\brak{\mbf{x}}$}.
\label{fig.2.1}	
\end{figure}
%
\item
Obtain a theoretical solution for problem \ref{convex_code} using coordinate geometry.

\solution 
From \eqref{eq2_1_line} and \eqref{eq2_1_circ}, 
%
\begin{align}
r^2 & = (x_1-8)^2 + (3- x_1)^2 \\
&= 2 x_1^2 - 22 x_1 + 73 \\
\Rightarrow r^2 &= \frac{\brak{2x_1-11}^2 + 5^2}{2}
\end{align}
%
which is minium when $x_1 = \frac{11}{2}, x_2 = \frac{7}{2}$.  The minimum value is $\frac{25}{2}$ and 
the radius $r = \frac{5}{\sqrt{2}}$.
\item
\label{lagrange}
	Define 
	\begin{equation}
	\label{lagrangian}
	L\brak{\mbf{x},\lambda} = f\brak{\mbf{x}} - \lambda g\brak{\mbf{x}}%, \quad \lambda > 0
	\end{equation}
and
\begin{equation}
\nabla =  
\begin{pmatrix}
\frac{\partial}{\partial x_1} \\
\frac{\partial}{\partial x_2} \\
\frac{\partial}{\partial \lambda} 
\end{pmatrix}
\end{equation}
Solve the equations
%
\begin{align}
\nabla L\brak{\mbf{x},\lambda} &= 0.
\label{tangent}
\end{align}
%
How is this related to problem \ref{convex_code}? What is the sign of $\lambda$?  $L$ is known as the Lagrangian and the above technique is known as the Method of Lagrange Multipliers.

\solution
From \eqref{eq2_1_line} and \eqref{eq2_1_circ}, 
%
\begin{align}
L\brak{\mbf{x},\lambda} &= (x_1-8)^2 + (x_2-6)^2 - \lambda \brak{x_1 + x_2 - 9} \\
\Rightarrow \nabla L\brak{\mbf{x},\lambda}  & = 
\begin{pmatrix}
2x_1  - 16 - \lambda \\
2x_2 - 12 - \lambda \\
x_1 + x_2 -9
\end{pmatrix}
\\
&=
\begin{pmatrix}
2 &0 & - 1 \\
0 &2 & - 1 \\
1 & 1 & 0 
\end{pmatrix}
\begin{pmatrix}
x_1 \\
x_2 \\
\lambda
\end{pmatrix}
= 
\begin{pmatrix}
16 \\
 12 \\
9
\end{pmatrix}
=
0 
\\
\Rightarrow 
\begin{pmatrix}
x_1 \\
x_2 \\
\lambda
\end{pmatrix}
&= 
\begin{pmatrix}
\frac{11}{2} \\
 \frac{7}{2} \\
-5
\end{pmatrix}
\end{align}
%
using the following python script.  Note that this method yields the same result as the previous exercises.  Thus, $\lambda$ is negative.
%	
\begin{lstlisting}
wget https://raw.githubusercontent.com/gadepall/EE2250/master/manual/codes/2.3.py
\end{lstlisting}

%
\item
\label{ch2_constraint}
Modify the code in problem \ref{convex_code} to find a graphical solution for minimising
\begin{align}
f\brak{\mbf{x}} = (x_1-8)^2 + (x_2-6)^2
\end{align}
with constraint
\begin{align}
%\label{convex-constraint}
g\brak{\mbf{x}} = x_1 + x_2 - 9 \geq 0
\end{align}

\solution 
This problem reduces to finding the radius of the smallest circle in the shaded area in Fig. \ref{fig.2.4} .  It is clear that this radius is 0.
%	
\begin{lstlisting}
wget https://raw.githubusercontent.com/gadepall/EE2250/master/manual/codes/2.4.py
\end{lstlisting}

%
\begin{figure}[!ht]
\centering
\includegraphics[width=\columnwidth]{./figs/2.4.eps}
\caption{ Smallest circle in the shaded region is a point.}
\label{fig.2.4}	
\end{figure}
%
\end{enumerate}
\section{Karush Kuhn-Tucker Conditions}
\begin{enumerate}[label=\thesection.\arabic*,ref=\thesection.\theenumi]

\item
\label{ch2_lagrange_fail}
Now use the method of Lagrange multipliers to solve  problem \ref{ch2_constraint} and compare with the graphical solution.  Comment.

%
\solution Using the method of Lagrange multipliers, the solution is the same as the one obtained in  problem \ref{ch2_constraint}, which is different from the graphical solution.  This means that the Lagrange multipliers method cannot be applied blindly.
\item
Repeat problem \ref{ch2_lagrange_fail} by keeping 
 $\lambda=0$.   Comment.

\solution Keeping $\lambda = 0$ results in $x_1 = 8, x_2 = 6$, which is the correct solution.  The minimum value of $f\brak{\mbf{x}}$ without any constraints lies in the region $g\brak{\mbf{x}} = 0$.  In this case, $\lambda = 0$.  
%
%
\item
\label{ch2_constraint_border}
Find a graphical solution for minimising
\begin{align}
f\brak{\mbf{x}} = (x_1-8)^2 + (x_2-6)^2
\end{align}
with constraint
\begin{align}
%\label{convex-constraint}
g\brak{\mbf{x}} = x_1 + x_2 - 9 \leq 0.
\end{align}
Summarize your observations.

%
\solution In Fig. \ref{fig.2.7}, the shaded region represents the constraint.  Thus, the solution is the same as the one in problem \ref{ch2_constraint}. This implies that the method of
Lagrange multipliers can be used to solve the optimization problem with this inequality constraint as well.  Table \ref{table.2.7} summarizes the conditions for this based on the observations so far.
\begin{lstlisting}
wget https://raw.githubusercontent.com/gadepall/EE2250/master/manual/codes/2.7.py
\end{lstlisting}

%
\begin{figure}[!ht]
\centering
\includegraphics[width=\columnwidth]{./figs/2.7.eps}
\caption{ Finding $ \displaystyle \min_{\mbf{x}}f\brak{\mbf{x}}$.}
\label{fig.2.7}	
\end{figure}
\input{./figs/tab.2.7.tex}
%
\item
\label{ch2_prob_upper}
Find a graphical solution for 	 
	 \begin{align}
	 \label{ch2_second_min}
	\min_{\mbf{x}} f\brak{\mbf{x}} = (x_1-8)^2 + (x_2-6)^2
	 \end{align}
	 with constraint
	 \begin{align}
	 \label{ch2_second_const}
	 g\brak{\mbf{x}} = x_1 + x_2 - 18 = 0
	 \end{align}
	 
%
\solution
%	
\begin{lstlisting}
wget https://raw.githubusercontent.com/gadepall/EE2250/master/manual/codes/2.8.py
\end{lstlisting}

%
\begin{figure}[!ht]
\centering
\includegraphics[width=\columnwidth]{./figs/2.8.eps}
\caption{ Finding $ \displaystyle \min_{\mbf{x}}f\brak{\mbf{x}}$.}
\label{fig.2.8}	
\end{figure}
%
\item
Repeat problem \ref{ch2_prob_upper} using the method of Lagrange mutipliers.  What is the sign of $\lambda$?

%
\solution
From \eqref{ch2_second_min} and \eqref{ch2_second_const}, 
%
\begin{align}
L\brak{\mbf{x},\lambda} &= (x_1-8)^2 + (x_2-6)^2 - \lambda \brak{x_1 + x_2 - 18} \\
\Rightarrow \nabla L\brak{\mbf{x},\lambda}  & = 
\begin{pmatrix}
2x_1  - 16 - \lambda \\
2x_2 - 12 - \lambda \\
x_1 + x_2 -18
\end{pmatrix}
\\
&=
\begin{pmatrix}
2 &0 & - 1 \\
0 &2 & - 1 \\
1 & 1 & 0 
\end{pmatrix}
\begin{pmatrix}
x_1 \\
x_2 \\
\lambda
\end{pmatrix}
= 
\begin{pmatrix}
16 \\
 12 \\
18
\end{pmatrix}
=
0 
\\
\Rightarrow 
\begin{pmatrix}
x_1 \\
x_2 \\
\lambda
\end{pmatrix}
&= 
\begin{pmatrix}
10 \\
 8 \\
4
\end{pmatrix}
\end{align}
%
using the following python script.  Thus, $\lambda$ is positive and the minimum value of $f$ is 8.
%	
\begin{lstlisting}
wget https://raw.githubusercontent.com/gadepall/EE2250/master/manual/codes/2.9.py
\end{lstlisting}

%
%
\item
\label{ch2_prob_upper_cond}
Solve
	 \begin{align}
%	 \label{ch2_second_min}
	\min_{\mbf{x}} f\brak{\mbf{x}} = (x_1-8)^2 + (x_2-6)^2
	 \end{align}
	 with constraint
	 \begin{align}
%	 \label{ch2_second_const}
	 g\brak{\mbf{x}} = x_1 + x_2 - 18 \geq 0 
	 \end{align}
	 
%
\solution Since the unconstrained solution is outside the region $g\brak{\mbf{x}} \geq 0$, the solution is the same as the one in problem \ref{ch2_prob_upper}.
%
\item
Based on the problems so far, generalise the Lagrange multipliers method for 
%
	 \begin{align}
	 \label{ch2_lagrange_ineq}
	\min_{\mbf{x}} f\brak{\mbf{x}} , \quad 
	 g\brak{\mbf{x}}  \geq 0 
	 \end{align}
%

%
\solution
Considering $L\brak{\mbf{x},\lambda} = f\brak{\mbf{x}} - \lambda g\brak{\mbf{x}}$, for $g\brak{\mbf{x}} = x_1 + x_2 - 18 \geq 0$ we found $\lambda > 0 $ and for $g\brak{\mbf{x}} = x_1 + x_2 - 9 \leq 0, \lambda < 0$. A single condition can be obtained by framing the optimization problem as
%
	 \begin{align}
	 \label{ch2_lagrange_ineq_summary}
	\min_{\mbf{x}} f\brak{\mbf{x}} , \quad 
	 g\brak{\mbf{x}}  \leq 0 
	 \end{align}
%
with the Lagrangian
%
\begin{equation}
%\label{ch2_kkt_necessary}
L\brak{\mbf{x},\lambda} = f\brak{\mbf{x}} + \lambda g\brak{\mbf{x}}, %\quad  \lambda > 0,  g\brak{\mbf{x}} \leq 0.
\end{equation}
%
provided
%
\begin{equation}
\label{ch2_kkt_necessary}
\nabla L\brak{\mbf{x},\lambda} = 0 \Rightarrow \lambda > 0
\end{equation}
else, $\lambda = 0$.
\item
Solve
 \begin{align}
 \label{ch2_kkt_problem}
\min_{\mbf{x}} f\brak{\mbf{x}} = 4x_1^2 + 2x_2^2
 \end{align}
 with constraints
 \begin{align}
 g_1\brak{\mbf{x}} = 3x_1 + x_2-8 = 0\\
 g_2 \brak{\mbf{x}}= 15 - 2x_1 - 4x_2 \geq 0
 \end{align}
 
%
\solution Considering the Lagrangian
%
\begin{align}
L\brak{\mbf{x},\lambda} &= f\brak{\mbf{x}} + \lambda g_1\brak{\mbf{x}} - \mu g_2\brak{\mbf{x}} \\
 &= 4x_1^2 + 2x_2^2 + \lambda \brak{3x_1 + x_2-8} 
 \nonumber \\
 &\,-\mu\brak{15 - 2x_1 - 4x_2},\\
 \nabla L\brak{\mbf{x},\lambda}  & = 
\begin{pmatrix}
8x_1 + 3 \lambda  +2 \mu  \\
4x_2 + \lambda + 4 \mu \\
3x_1 + x_2 -8 \\
 - 2x_1 - 4x_2 + 15
\end{pmatrix}
= 0
\end{align}
%
resulting in the matrix equation
%
\begin{align}
\Rightarrow 
\begin{pmatrix}
8 &0 & 3 & 2\\
0 &4 & 1 & 4 \\
3 & 1 & 0 &0  \\
2 & 4 & 0 & 0
\end{pmatrix}
\begin{pmatrix}
x_1 \\
x_2 \\
\lambda
\\
\mu
\end{pmatrix}
&=
\begin{pmatrix}
0 \\
0 \\
8 \\
15
\end{pmatrix}
\\
\Rightarrow 
\begin{pmatrix}
x_1 \\
x_2 \\
\lambda
\\
\mu
\end{pmatrix}
&= 
\begin{pmatrix}
1.7 \\
 2.9 \\
-3.12 \\
-2.12
\end{pmatrix}
\end{align}
%
using the following python script.  The (incorrect) graphical solution is available in Fig. \ref{fig.2.12}
%	
\begin{lstlisting}
wget https://raw.githubusercontent.com/gadepall/EE2250/master/manual/codes/2.12.py
\end{lstlisting}

%
Note that $\mu < 0 $, contradicting the necessary condition in \eqref{ch2_kkt_necessary}. 
%
\begin{figure}[!ht]
\centering
\includegraphics[width=\columnwidth]{./figs/2.12_1.eps}
\caption{ Incorrect solution is at intersection of all curves $r = 5.33$}
\label{fig.2.12}	
\end{figure}
\item
Obtain the correct solution to the previous problem by considering $\mu = 0$.

\begin{figure}[!ht]
\centering
\includegraphics[width=\columnwidth]{./figs/2.12_2.eps}
\caption{ Optimal solution is where $g_1(x)$ touches the curve $r = 4.82$}
\label{fig.2.13}	
\end{figure}
%
%
\item
Solve
 \begin{align}
% \label{ch2_kkt_problem}
\min_{\mbf{x}} f\brak{\mbf{x}} = 4x_1^2 + 2x_2^2
 \end{align}
 with constraints
 \begin{align}
 g_1\brak{\mbf{x}} = 3x_1 + x_2-8 = 0\\
 g_2 \brak{\mbf{x}}= 15 - 2x_1 - 4x_2 \leq 0
 \end{align}
 
%
\item
Based on whatever you have done so far,	list the steps that you would use in general for solving a convex optimization problem  like \eqref{ch2_kkt_problem}  using Lagrange Multipliers. 
These are called Karush-Kuhn-Tucker(KKT) conditions.

\solution For a problem defined by 
\begin{align}
\mbf{x^*} &= \min_{\mbf{x}}f(\mbf{x})
\\
Primal\text{ } Feasibility:
\nonumber\\
\text{subject to } h_i(\mbf{x}) &= 0, \forall i=1,..,m
\\
\text{subject to } g_i(\mbf{x}) &\le 0, \forall i=1,..,n
\end{align}
%
the optimal solution is obtained through
%
\begin{align}
\mbf{x^*} &= \min_{\mbf{x}}L(\mbf{x}, \mbf{\lambda}, \mbf{\mu}) 
\\
&= \min_{\mbf{x}}f(\mbf{x})  + \underset{i=1}{\overset{m}{\sum}} \lambda_i h_i(\mbf{x}) + \underset{i=1}{\overset{n}{\sum}} \mu_i g_i(\mbf{x}),
\end{align}
%
using the KKT conditions
%
\begin{align}
Stationarity:
\nonumber\\
\Rightarrow \nabla_\mbf{x} f(\mbf{x})  + \underset{i=1}{\overset{m}{\sum}}  \lambda_i 
\nabla_\mbf{x}h_i(\mbf{x}) + \underset{i=1}{\overset{n}{\sum}} \mu_i \nabla_\mbf{x} g_i(\mbf{x}) = 0 
\\
Complementary \text{ } Slackness:
\nonumber\\
\text{subject to }\mu_i g_i(\mbf{x}) = 0, \forall i = 1,..,n
\\
Dual\text{ } Feasibility:
\nonumber\\
\text{and }\mu_i \ge 0, \forall i = 1,..,n
\end{align}
%
\item
	Maxmimize 
	%
	\begin{align}
	f(\mbf{x}) &= \sqrt{x_1x_2}
	\end{align}
	%
	with the constraints
	%
	\begin{align}
	x_1^2&+x_2^2 &\leq 5 \\
	x_1 \geq 0,& x_2 &\geq 0
	\end{align}
	%

%
\item
	\label{convex_sdp_eqiv}
	%
	Solve
	\begin{equation}
	\min_{\mbf{x}} \quad x_1 + x_2
	\end{equation}
	%	
	with the constraints
	\begin{equation}
	x_1^2 - x_1 + x_2^2 \leq 0
	\end{equation}
	%
where 
$
\mbf{x} = \begin{pmatrix}
x_1 \\
x_2
\end{pmatrix}
$

\solution Using the method of Lagrange multipliers,
%
\begin{align}
\label{ch2_sd_kkt}
\nabla \cbrak{f(\mbf{x})  +  \mu g(\mbf{x}) }= 0 , \quad \mu \ge 0
\end{align}
%
resulting in the equations
%
\begin{align}
2x_1\mu -\mu + 1 &= 0 \\
2x_2\mu + 1 &=0 \\
x_1^2 -x_1 + x_2^2 &= 0 
\end{align}
%
which can be simplified to obtain 
%
\begin{align}
\brak{\frac{1-\mu}{2\mu}}^2 + \brak{\frac{1}{2\mu}}^2 + \frac{1-\mu}{2\mu} &= 0 \\
\Rightarrow 1 + \mu^2 -2\mu + 1 + 2\mu\brak{1-\mu} &= 0 \\
\Rightarrow \mu^2 =2, or \mu &= \pm \sqrt{2} 
\end{align}
%
From \eqref{ch2_kkt_problem},  $\mu \ge 0 \Rightarrow  \mu = \sqrt{2}$. The desired solution is
%
\begin{equation}
\mbf{x} = 
\begin{pmatrix}
 \frac{\sqrt{2}-1}{2\sqrt{2}} \\
-\frac{1}{2\sqrt{2}} 
\end{pmatrix}
\end{equation}
%
\\
{\em Graphical solution:} The constraint can be expressed as
%
\begin{align}
x_1^2 - x_1 + x_2^2 &\le 0 \\
\Rightarrow \brak{x_1 - \frac{1}{2}}^2 + x_2^2 & \le \brak{\frac{1}{2}}^2
\end{align}
%
%	
\begin{lstlisting}
wget https://raw.githubusercontent.com/gadepall/EE2250/master/manual/codes/2.15.py
\end{lstlisting}

%
%
\begin{figure}[!ht]
\centering
\includegraphics[width=\columnwidth]{./figs/2.15.eps}
\caption{ Optimal solution is the lower tangent to the circle}
\label{fig.2.15}	
\end{figure}
%
\end{enumerate}

	
\end{document}

