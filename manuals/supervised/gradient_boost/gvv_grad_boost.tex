\documentclass[journal,12pt,twocolumn]{IEEEtran}
\usepackage{setspace}
\usepackage{gensymb}
\usepackage{caption}
%\usepackage{multirow}
%\usepackage{multicolumn}
%\usepackage{subcaption}
%\doublespacing
\singlespacing
\usepackage{csvsimple}
\usepackage{amsmath}
\usepackage{multicol}
%\usepackage{enumerate}
\usepackage{amssymb}
%\usepackage{graphicx}
\usepackage{newfloat}
%\usepackage{syntax}
\usepackage{listings}
\usepackage{iithtlc}
\usepackage{color}
\usepackage{tikz}
\usetikzlibrary{shapes,arrows}



%\usepackage{graphicx}
%\usepackage{amssymb}
%\usepackage{relsize}
%\usepackage[cmex10]{amsmath}
%\usepackage{mathtools}
%\usepackage{amsthm}
%\interdisplaylinepenalty=2500
%\savesymbol{iint}
%\usepackage{txfonts}
%\restoresymbol{TXF}{iint}
%\usepackage{wasysym}
\usepackage{amsthm}
\usepackage{mathrsfs}
\usepackage{txfonts}
\usepackage{stfloats}
\usepackage{cite}
\usepackage{cases}
\usepackage{mathtools}
\usepackage{caption}
\usepackage{enumerate}	
\usepackage{enumitem}
\usepackage{amsmath}
%\usepackage{xtab}
\usepackage{longtable}
\usepackage{multirow}
%\usepackage{algorithm}
%\usepackage{algpseudocode}
\usepackage{enumitem}
\usepackage{mathtools}
\usepackage{hyperref}
%\usepackage[framemethod=tikz]{mdframed}
\usepackage{listings}
    %\usepackage[latin1]{inputenc}                                 %%
    \usepackage{color}                                            %%
    \usepackage{array}                                            %%
    \usepackage{longtable}                                        %%
    \usepackage{calc}                                             %%
    \usepackage{multirow}                                         %%
    \usepackage{hhline}                                           %%
    \usepackage{ifthen}                                           %%
  %optionally (for landscape tables embedded in another document): %%
    \usepackage{lscape}     


\usepackage{url}
\def\UrlBreaks{\do\/\do-}


%\usepackage{stmaryrd}


%\usepackage{wasysym}
%\newcounter{MYtempeqncnt}
\DeclareMathOperator*{\Res}{Res}
%\renewcommand{\baselinestretch}{2}
\renewcommand\thesection{\arabic{section}}
\renewcommand\thesubsection{\thesection.\arabic{subsection}}
\renewcommand\thesubsubsection{\thesubsection.\arabic{subsubsection}}

\renewcommand\thesectiondis{\arabic{section}}
\renewcommand\thesubsectiondis{\thesectiondis.\arabic{subsection}}
\renewcommand\thesubsubsectiondis{\thesubsectiondis.\arabic{subsubsection}}

% correct bad hyphenation here
\hyphenation{op-tical net-works semi-conduc-tor}

%\lstset{
%language=C,
%frame=single, 
%breaklines=true
%}

%\lstset{
	%%basicstyle=\small\ttfamily\bfseries,
	%%numberstyle=\small\ttfamily,
	%language=Octave,
	%backgroundcolor=\color{white},
	%%frame=single,
	%%keywordstyle=\bfseries,
	%%breaklines=true,
	%%showstringspaces=false,
	%%xleftmargin=-10mm,
	%%aboveskip=-1mm,
	%%belowskip=0mm
%}

%\surroundwithmdframed[width=\columnwidth]{lstlisting}
\def\inputGnumericTable{}                                 %%
\lstset{
%language=C,
frame=single, 
breaklines=true,
columns=fullflexible
}
 

\begin{document}
%
\tikzstyle{block} = [rectangle, draw,
    text width=3em, text centered, minimum height=3em]
\tikzstyle{sum} = [draw, circle, node distance=3cm]
\tikzstyle{input} = [coordinate]
\tikzstyle{output} = [coordinate]
\tikzstyle{pinstyle} = [pin edge={to-,thin,black}]

\theoremstyle{definition}
\newtheorem{theorem}{Theorem}[section]
\newtheorem{problem}{Problem}
\newtheorem{proposition}{Proposition}[section]
\newtheorem{lemma}{Lemma}[section]
\newtheorem{corollary}[theorem]{Corollary}
\newtheorem{example}{Example}[section]
\newtheorem{definition}{Definition}[section]
%\newtheorem{algorithm}{Algorithm}[section]
%\newtheorem{cor}{Corollary}
\newcommand{\BEQA}{\begin{eqnarray}}
\newcommand{\EEQA}{\end{eqnarray}}
\newcommand{\define}{\stackrel{\triangle}{=}}

\bibliographystyle{IEEEtran}
%\bibliographystyle{ieeetr}

\providecommand{\nCr}[2]{\,^{#1}C_{#2}} % nCr
\providecommand{\nPr}[2]{\,^{#1}P_{#2}} % nPr
\providecommand{\mbf}{\mathbf}
\providecommand{\pr}[1]{\ensuremath{\Pr\left(#1\right)}}
\providecommand{\qfunc}[1]{\ensuremath{Q\left(#1\right)}}
\providecommand{\sbrak}[1]{\ensuremath{{}\left[#1\right]}}
\providecommand{\lsbrak}[1]{\ensuremath{{}\left[#1\right.}}
\providecommand{\rsbrak}[1]{\ensuremath{{}\left.#1\right]}}
\providecommand{\brak}[1]{\ensuremath{\left(#1\right)}}
\providecommand{\lbrak}[1]{\ensuremath{\left(#1\right.}}
\providecommand{\rbrak}[1]{\ensuremath{\left.#1\right)}}
\providecommand{\cbrak}[1]{\ensuremath{\left\{#1\right\}}}
\providecommand{\lcbrak}[1]{\ensuremath{\left\{#1\right.}}
\providecommand{\rcbrak}[1]{\ensuremath{\left.#1\right\}}}
\theoremstyle{remark}
\newtheorem{rem}{Remark}
\newcommand{\sgn}{\mathop{\mathrm{sgn}}}
\providecommand{\abs}[1]{\left\vert#1\right\vert}
\providecommand{\res}[1]{\Res\displaylimits_{#1}} 
\providecommand{\norm}[1]{\lVert#1\rVert}
\providecommand{\mtx}[1]{\mathbf{#1}}
\providecommand{\mean}[1]{E\left[ #1 \right]}
\providecommand{\fourier}{\overset{\mathcal{F}}{ \rightleftharpoons}}
%\providecommand{\hilbert}{\overset{\mathcal{H}}{ \rightleftharpoons}}
\providecommand{\system}{\overset{\mathcal{H}}{ \longleftrightarrow}}
	%\newcommand{\solution}[2]{\textbf{Solution:}{#1}}
\newcommand{\solution}{\noindent \textbf{Solution: }}
\providecommand{\dec}[2]{\ensuremath{\overset{#1}{\underset{#2}{\gtrless}}}}
\DeclarePairedDelimiter{\ceil}{\lceil}{\rceil}
%\numberwithin{equation}{subsection}
\numberwithin{equation}{section}
%\numberwithin{problem}{subsection}
%\numberwithin{definition}{subsection}
\makeatletter
\@addtoreset{figure}{section}
\makeatother

\let\StandardTheFigure\thefigure
%\renewcommand{\thefigure}{\theproblem.\arabic{figure}}
\renewcommand{\thefigure}{\thesection}


%\numberwithin{figure}{subsection}

%\numberwithin{equation}{subsection}
%\numberwithin{equation}{section}
%\numberwithin{equation}{problem}
%\numberwithin{problem}{subsection}
\numberwithin{problem}{section}
%%\numberwithin{definition}{subsection}
%\makeatletter
%\@addtoreset{figure}{problem}
%\makeatother
\makeatletter
\@addtoreset{table}{section}
\makeatother

\let\StandardTheFigure\thefigure
\let\StandardTheTable\thetable
%%\renewcommand{\thefigure}{\theproblem.\arabic{figure}}
%\renewcommand{\thefigure}{\theproblem}

%%\numberwithin{figure}{section}

%%\numberwithin{figure}{subsection}



\def\putbox#1#2#3{\makebox[0in][l]{\makebox[#1][l]{}\raisebox{\baselineskip}[0in][0in]{\raisebox{#2}[0in][0in]{#3}}}}
     \def\rightbox#1{\makebox[0in][r]{#1}}
     \def\centbox#1{\makebox[0in]{#1}}
     \def\topbox#1{\raisebox{-\baselineskip}[0in][0in]{#1}}
     \def\midbox#1{\raisebox{-0.5\baselineskip}[0in][0in]{#1}}

\vspace{3cm}

\title{ 
	\logo{
Gradient Boost: Least Mean Square Algorithm
	}
}

\author{B Swaroop Reddy and G V V Sharma$^{*}$% <-this % stops a space
	\thanks{*The author is with the Department
		of Electrical Engineering, Indian Institute of Technology, Hyderabad
		502285 India e-mail:  gadepall@iith.ac.in. All content in this manual is released under GNU GPL.  Free and open source.}
	
}	

\maketitle

\tableofcontents

\bigskip

\renewcommand{\thefigure}{\theenumi}
\renewcommand{\thetable}{\theenumi}


\begin{abstract}
	
This manual provides an introduction to the LMS algorithm.
		
\end{abstract}
\section{Convex Functions}

%\subsection{Problem}
A single variable function $f$ is said to be convex if
%
\begin{align}
\label{ch1_convex_def}
f\sbrak{\lambda x + \brak{1-\lambda}y} \leq \lambda f\brak{x} + \brak{1-\lambda}f\brak{y}, 
\end{align}
%
for $\quad 0 < \lambda < 1$.
\begin{enumerate}[label=\thesection.\arabic*,ref=\thesection.\theenumi]

\item
Download and execute the following python script. Is  $\ln x$ convex or  concave?

%
\begin{lstlisting}[language=sh]
https://raw.githubusercontent.com/gadepall/EE1390/master/manuals/supervised/gradient_boost/codes/1.1.py
\end{lstlisting}
%
\begin{figure}[!ht]
\centering
\includegraphics[width=\columnwidth]{./figs/1.1.eps}
\caption{ $\ln x$ versus $x$}.
\label{fig.1.1}	
\end{figure}
%
\item
Modify the above python script as follows to plot the parabola $f(x) = x^2$. Is it convex or concave?

\begin{lstlisting}
https://raw.githubusercontent.com/gadepall/EE1390/master/manuals/supervised/gradient_boost/codes/1.2.py
\end{lstlisting}
%
\begin{figure}[!ht]
\centering
\includegraphics[width=\columnwidth]{./figs/1.2.eps}
\caption{ $x^2$ versus $x$}.
\label{fig.1.2}	
\end{figure}
%
\item
Execute the following script to obtain Fig. \ref{fig.1.3}. Comment.

%
\begin{lstlisting}
https://raw.githubusercontent.com/gadepall/EE1390//master/manuals/supervised/gradient_boost/codes/1.3.py
\end{lstlisting}

%
\begin{figure}[!ht]
\centering
\includegraphics[width=\columnwidth]{./figs/1.3.eps}
\caption{ Segments are below the curve}.
\label{fig.1.3}	
\end{figure}
%
\item
Modify the script in the previous problem for $f(x) = x^2$.  What can you conclude?

\item
Let 
\begin{equation}
f(\mathbf{x}) = x_1x_2, \quad \mathbf{x} \in \mathbf{R}^2
\end{equation}
Sketch $f(\mathbf{x})$ and deduce whether it is convex.
Can you theoretically explain your observation using \eqref{ch1_convex_def}?
\end{enumerate}
%
\section{Gradient Descent Method}
Consider the problem of finding the square root of a number $c$.  This can be expressed as the equation
%
\begin{equation}
\label{eq:root}
x^2 -c= 0
\end{equation}
%
\begin{enumerate}[label=\thesection.\arabic*,ref=\thesection.\theenumi]

\item
Sketch the function for different values of $c$
%
\begin{equation}
f(x)= x^{3}-3xc
\end{equation}
%
and comment upon its convexity.

\item
Show that \eqref{eq:root} results from
\begin{align}
\min_{x}f(x)= x^{3}-3xc
\end{align}

\item
Find a numerical solution for \eqref{eq:root}.

\solution
A numerical solution for \eqref{eq:root} is obtained as
%
\begin{align}
x_{n+1}&=x_{n}-\mu f^{\prime}\brak{x}
\\
&=x_{n} -\mu \brak{3x_n^{2}-3c}
\label{eq:gradient}
\end{align}

%\begin{align}
%x_{n+1}&=x_{n}-{\frac {f(x_{n})}{f^{\prime}(x_{n})}}
%\\
%&=x_{n} -\frac{x^2_{n}-c}{2x_n} 
%\\
%&=\frac{1}{2}\sbrak{x_{n} +\frac{c}{x_n} }
%\label{eq:newton}
%\end{align}
%
where $x_0$ is an inital guess.
%
\item
Write a program to implement \eqref{eq:gradient}.

%
\solution Download and execute
\begin{lstlisting}
wget 
https://raw.githubusercontent.com/gadepall/EE1390/master/manuals/supervised/gradient_boost/codes/square_root.py
\end{lstlisting}
\end{enumerate}

\section{Audio Source Files}
\begin{enumerate}[label=\thesection.\arabic*
,ref=\thesection.\theenumi]
%%
\item Get the \textbf{audio\_source}
\begin{lstlisting}
svn checkout https://github.com/gadepall/EE5347/trunk/audio_source
cd audio_source
\end{lstlisting}
\item Play the \textbf{signal\_noise.wav} and \textbf{noise.wav} file. Comment.
\\
\solution
\textbf{signal\_noise.wav}  contains a human voice along 
with an instrument sound in the background.  This instrument sound
is captured in \textbf{noise.wav}.
\end{enumerate}
%
\section{Problem Formulation}
\begin{enumerate}[label=\thesection.\arabic*
,ref=\thesection.\theenumi]

\item See Table \ref{table:known}.  The goal is to extract the human voice $e(n)$ from $d(n)$ by suppressing the component of $\mbf{X}(n)$.  Formulate 
an equation for this.
\begin{table}[!ht]
\centering
\small
\input{./figs/sigs.tex}
\caption{}
\label{table:known}
\end{table}
%
\solution The  maximum component of $\mbf{X}(n)$ in $d(n)$ can be estimated as
%
\begin{equation}
\label{eq:component}
\mbf{W}^{T}(n)\mbf{X}(n)
\end{equation}
%		where $e(n)$ is an estimate of the human voice ( desired signal) 
where 
\begin{align}
 \mbf{W}(n)
 =
 %\frac{1}{\det(X)}
  \begin{bmatrix}
   w_1(n) \\ w_2(n)\\
   w_3(n) \\ $..$ \\ $..$ \\ w_{n-M+1}(n)  \end{bmatrix}_{M X 1}
\end{align}
%
Intuitively, the human voice $e(n)$ is obtained after removing as much of $\mbf{X}(n)$ from $d(n)$ as 
possible. The first step in this direction is to estimate $\mbf{W}$ in \eqref{eq:component} using the metric
\begin{equation}
\label{eq:prob_mse}
\min_{\mbf{W}(n)}\, \norm{d(n)-\mbf{W}^{T}(n)\mbf{X}(n)}^2
\end{equation}
%
The human voice can be then obtained as
%
\begin{align}
\label{eq:error}
e(n) = d(n)-\mbf{W}^{T}(n)\mbf{X}(n)
\end{align}
%
%
\end{enumerate}
\section{LMS Algorithm}
%
\begin{enumerate}[label=\thesection.\arabic*
,ref=\thesection.\theenumi]

%\begin{enumerate}
%\item
%Show that $e^2(n)$ is a convex function.
%
\item
Show using \eqref{eq:error}  that 
\begin{align}
\nabla_{\mbf{W}(n)}e^2(n)&=\frac{\partial e^{2}(n)}{\partial \mbf{W}(n)}\\
&=- 2\mbf{X}(n)d(n) + 2 \mbf{X}(n) X^{T}(n)\mbf{W}(n)
\end{align}
%
\item
Use the gradient descent method to obtain an algorithm for solving \eqref{eq:prob_mse}

\solution The desired algorithm can be expressed as
%
\begin{align}
\mbf{W}(n+1)&=\mbf{W}(n) - \bar{\mu}[ \nabla_{\mbf{W}(n)}e^2(n)]
\\
\mbf{W}(n+1)&=\mbf{W}(n)+ \mu \mbf{X}(n) e(n)
\end{align}
%
where $\mu = \bar{\mu}$.
\item
Write a program to suppress $\mbf{X}(n)$ in $d(n)$.

\solution Execute 
\begin{lstlisting}
https://raw.githubusercontent.com/gadepall/EE1390/master/manuals/supervised/gradient_boost/codes/LMS_NC_SPEECH.py
\end{lstlisting}

%\textbf{LMS\_NC\_SPEECH.py}.
\end{enumerate}



\end{document}
