\documentclass[journal,12pt,twocolumn]{IEEEtran}
%
\usepackage{setspace}
\usepackage{gensymb}
\usepackage{xcolor}
\usepackage{caption}
%\usepackage{subcaption}
%\doublespacing
\singlespacing

%\usepackage{graphicx}
%\usepackage{amssymb}
%\usepackage{relsize}
\usepackage[cmex10]{amsmath}
\usepackage{mathtools}
%\usepackage{amsthm}
%\interdisplaylinepenalty=2500
%\savesymbol{iint}
%\usepackage{txfonts}
%\restoresymbol{TXF}{iint}
%\usepackage{wasysym}
\usepackage{hyperref}
\usepackage{amsthm}
\usepackage{mathrsfs}
\usepackage{txfonts}
\usepackage{stfloats}
\usepackage{cite}
\usepackage{cases}
\usepackage{subfig}
%\usepackage{xtab}
\usepackage{longtable}
\usepackage{multirow}
%\usepackage{algorithm}
%\usepackage{algpseudocode}
%\usepackage{enumerate}
\usepackage{enumitem}
\usepackage{mathtools}
\usepackage{iithtlc}
%\usepackage[framemethod=tikz]{mdframed}
\usepackage{listings}


%\usepackage{stmaryrd}
        \def\inputGnumericTable{}                                 %%

    \usepackage[latin1]{inputenc}                                 %%
    \usepackage{color}                                            %%
    \usepackage{array}                                            %%
    \usepackage{longtable}                                        %%
    \usepackage{calc}                                             %%
    \usepackage{multirow}                                         %%
    \usepackage{hhline}                                           %%
    \usepackage{ifthen}                                           %%
    \usepackage{lscape}                                           %%


%\usepackage{wasysym}
%\newcounter{MYtempeqncnt}
\DeclareMathOperator*{\Res}{Res}
%\renewcommand{\baselinestretch}{2}
\renewcommand\thesection{\arabic{section}}
\renewcommand\thesubsection{\thesection.\arabic{subsection}}
\renewcommand\thesubsubsection{\thesubsection.\arabic{subsubsection}}

\renewcommand\thesectiondis{\arabic{section}}
\renewcommand\thesubsectiondis{\thesectiondis.\arabic{subsection}}
\renewcommand\thesubsubsectiondis{\thesubsectiondis.\arabic{subsubsection}}

%\renewcommand{\labelenumi}{\textbf{\theenumi}}
%\renewcommand{\theenumi}{P.\arabic{enumi}}

% correct bad hyphenation here
\hyphenation{op-tical net-works semi-conduc-tor}

\lstset{
language=Python,
frame=single, 
breaklines=true,
columns=fullflexible
}



\begin{document}
%

\theoremstyle{definition}
\newtheorem{theorem}{Theorem}[section]
\newtheorem{problem}{Problem}
\newtheorem{proposition}{Proposition}[section]
\newtheorem{lemma}{Lemma}[section]
\newtheorem{corollary}[theorem]{Corollary}
\newtheorem{example}{Example}[section]
\newtheorem{definition}{Definition}[section]
%\newtheorem{algorithm}{Algorithm}[section]
%\newtheorem{cor}{Corollary}
\newcommand{\BEQA}{\begin{eqnarray}}
\newcommand{\EEQA}{\end{eqnarray}}
\newcommand{\define}{\stackrel{\triangle}{=}}

\bibliographystyle{IEEEtran}
%\bibliographystyle{ieeetr}

\providecommand{\nCr}[2]{\,^{#1}C_{#2}} % nCr
\providecommand{\nPr}[2]{\,^{#1}P_{#2}} % nPr
\providecommand{\mbf}{\mathbf}
\providecommand{\pr}[1]{\ensuremath{\Pr\left(#1\right)}}
\providecommand{\qfunc}[1]{\ensuremath{Q\left(#1\right)}}
\providecommand{\sbrak}[1]{\ensuremath{{}\left[#1\right]}}
\providecommand{\lsbrak}[1]{\ensuremath{{}\left[#1\right.}}
\providecommand{\rsbrak}[1]{\ensuremath{{}\left.#1\right]}}
\providecommand{\brak}[1]{\ensuremath{\left(#1\right)}}
\providecommand{\lbrak}[1]{\ensuremath{\left(#1\right.}}
\providecommand{\rbrak}[1]{\ensuremath{\left.#1\right)}}
\providecommand{\cbrak}[1]{\ensuremath{\left\{#1\right\}}}
\providecommand{\lcbrak}[1]{\ensuremath{\left\{#1\right.}}
\providecommand{\rcbrak}[1]{\ensuremath{\left.#1\right\}}}
\theoremstyle{remark}
\newtheorem{rem}{Remark}
\newcommand{\sgn}{\mathop{\mathrm{sgn}}}
\providecommand{\abs}[1]{\left\vert#1\right\vert}
\providecommand{\res}[1]{\Res\displaylimits_{#1}} 
\providecommand{\norm}[1]{\lVert#1\rVert}
\providecommand{\mtx}[1]{\mathbf{#1}}
\providecommand{\mean}[1]{E\left[ #1 \right]}
\providecommand{\fourier}{\overset{\mathcal{F}}{ \rightleftharpoons}}
\providecommand{\ztrans}{\overset{\mathcal{Z}}{ \rightleftharpoons}}

%\providecommand{\hilbert}{\overset{\mathcal{H}}{ \rightleftharpoons}}
\providecommand{\system}{\overset{\mathcal{H}}{ \longleftrightarrow}}
	%\newcommand{\solution}[2]{\textbf{Solution:}{#1}}
\newcommand{\solution}{\noindent \textbf{Solution: }}
\providecommand{\dec}[2]{\ensuremath{\overset{#1}{\underset{#2}{\gtrless}}}}
\numberwithin{equation}{section}
%\numberwithin{equation}{subsection}
%\numberwithin{problem}{subsection}
%\numberwithin{definition}{subsection}
\makeatletter
\@addtoreset{figure}{problem}
\makeatother

\let\StandardTheFigure\thefigure
%\renewcommand{\thefigure}{\theproblem.\arabic{figure}}
\renewcommand{\thefigure}{\theproblem}



\def\putbox#1#2#3{\makebox[0in][l]{\makebox[#1][l]{}\raisebox{\baselineskip}[0in][0in]{\raisebox{#2}[0in][0in]{#3}}}}
     \def\rightbox#1{\makebox[0in][r]{#1}}
     \def\centbox#1{\makebox[0in]{#1}}
     \def\topbox#1{\raisebox{-\baselineskip}[0in][0in]{#1}}
     \def\midbox#1{\raisebox{-0.5\baselineskip}[0in][0in]{#1}}

\vspace{3cm}

\title{ 
	\logo{Toy Car as AI Bot
	}
}
\author{Raktim Gautam Goswami$^{1}$, Abhishek Bairagi$^{2}$, Alok Ranjan Kesari$^{3} 
$ \& G V V Sharma$^{4}$ 
%<-this  stops a space
\thanks{The authors are with the Department
of Electrical Engineering, Indian Institute of Technology, Hyderabad
502285 India .  e-mail: 1. ee17btech11004@iith.ac.in, 2. ee17btech11051@iith.ac.in, 3. arkesari@iith.ac.in, 4. gadepall@iith.ac.in}% <-this % stops a space
%%\thanks{J. Doe and J. Doe are with Anonymous University.}% <-this % stops a space
%%\thanks{Manuscript received April 19, 2005; revised January 11, 2007.}}
}




% make the title area
\maketitle

%\newpage

\tableofcontents

\renewcommand{\thefigure}{\theenumi}
\renewcommand{\thetable}{\theenumi}


\bigskip

\begin{abstract}
%
A Toycar is controlled using bluetooth and google voice recognition software.
%
\end{abstract}


\section{Hardware Setup}
%
\begin{enumerate}[label=\thesection.\arabic*
,ref=\thesection.\theenumi]

\item Assemble the motors, chassis and wheels to build the toycar.
%
\item Stick the breadboard to the chassis of the toycar.
%
	%
%

\item Stick a 9V battery to the breadboard and connect the positive and negative 
terminals to extreme ends of the breadboard.
%
\item 
Stick a 9V battery to the breadboard and connect the positive and negative terminals to extreme ends of the breadboard.
%
\item
Provide 9V to the supply pin of the Arduino.
%
\item
Plug the L293D motor driver IC in Fig. \ref{fig:l293d} on the breadboard.

	%
\begin{figure}[!ht]
\begin{center}
\includegraphics[width=\columnwidth]{./figs/l293d}
\end{center}
\caption{}
\label{fig:l293d}
\end{figure}
%
%
\item
Connect the L293D pins according to Table \ref{table:l293d}.

	\input{./figs/l293d.tex}
\item
Connect the HC05 pins according to Table \ref{table:hc05}.


\begin{figure}[ht!]
\begin{center}
\includegraphics[width=\columnwidth]{./figs/HC05}
\end{center}
\caption{HC05 Bluetooth module}
\label{fig:hc05}
\end{figure}

	\input{./figs/hc05.tex}


\end{enumerate}
%
\section{Implementation}
\begin{enumerate}[label=\thesection.\arabic*
,ref=\thesection.\theenumi]
\item Dump the following code in Arduino using its IDE.
\begin{lstlisting}
wget https://raw.githubusercontent.com/ gadepall/EE1390/master/bot/codes/vcb.cpp
\end{lstlisting}
%
\item Install Google API "Arduino Bluetooth Controller" using google play-store
\begin{figure}[!h]
\begin{center}
\includegraphics[width = \columnwidth]{./figs/app}
%\includegraphics[width = 7cm, height = 10cm]{./figs/app}
\end{center}
\caption{}
\label{fig:App}
\end{figure}
\item Open the app and connect to HC-05.
\item Open voice control section in the app and tap to give following commands. 

\textit{Left, Right, Forward, Back \& Stop}

\end{enumerate}

\end{document}


