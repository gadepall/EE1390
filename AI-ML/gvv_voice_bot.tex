\documentclass[journal,12pt,twocolumn]{IEEEtran}
%
\usepackage{setspace}
\usepackage{gensymb}
\usepackage{xcolor}
\usepackage{caption}
%\usepackage{subcaption}
%\doublespacing
\singlespacing

%\usepackage{graphicx}
%\usepackage{amssymb}
%\usepackage{relsize}
\usepackage[cmex10]{amsmath}
\usepackage{mathtools}
%\usepackage{amsthm}
%\interdisplaylinepenalty=2500
%\savesymbol{iint}
%\usepackage{txfonts}
%\restoresymbol{TXF}{iint}
%\usepackage{wasysym}
\usepackage{hyperref}
\usepackage{amsthm}
\usepackage{mathrsfs}
\usepackage{txfonts}
\usepackage{stfloats}
\usepackage{cite}
\usepackage{cases}
\usepackage{subfig}
%\usepackage{xtab}
\usepackage{longtable}
\usepackage{multirow}
%\usepackage{algorithm}
%\usepackage{algpseudocode}
%\usepackage{enumerate}
\usepackage{enumitem}
\usepackage{mathtools}
\usepackage{iithtlc}
%\usepackage[framemethod=tikz]{mdframed}
\usepackage{listings}


%\usepackage{stmaryrd}
        \def\inputGnumericTable{}                                 %%

    \usepackage[latin1]{inputenc}                                 %%
    \usepackage{color}                                            %%
    \usepackage{array}                                            %%
    \usepackage{longtable}                                        %%
    \usepackage{calc}                                             %%
    \usepackage{multirow}                                         %%
    \usepackage{hhline}                                           %%
    \usepackage{ifthen}                                           %%
    \usepackage{lscape}                                           %%


%\usepackage{wasysym}
%\newcounter{MYtempeqncnt}
\DeclareMathOperator*{\Res}{Res}
%\renewcommand{\baselinestretch}{2}
\renewcommand\thesection{\arabic{section}}
\renewcommand\thesubsection{\thesection.\arabic{subsection}}
\renewcommand\thesubsubsection{\thesubsection.\arabic{subsubsection}}

\renewcommand\thesectiondis{\arabic{section}}
\renewcommand\thesubsectiondis{\thesectiondis.\arabic{subsection}}
\renewcommand\thesubsubsectiondis{\thesubsectiondis.\arabic{subsubsection}}

%\renewcommand{\labelenumi}{\textbf{\theenumi}}
%\renewcommand{\theenumi}{P.\arabic{enumi}}

% correct bad hyphenation here
\hyphenation{op-tical net-works semi-conduc-tor}

\lstset{
language=Python,
frame=single, 
breaklines=true,
columns=fullflexible
}



\begin{document}
%

\theoremstyle{definition}
\newtheorem{theorem}{Theorem}[section]
\newtheorem{problem}{Problem}
\newtheorem{proposition}{Proposition}[section]
\newtheorem{lemma}{Lemma}[section]
\newtheorem{corollary}[theorem]{Corollary}
\newtheorem{example}{Example}[section]
\newtheorem{definition}{Definition}[section]
%\newtheorem{algorithm}{Algorithm}[section]
%\newtheorem{cor}{Corollary}
\newcommand{\BEQA}{\begin{eqnarray}}
\newcommand{\EEQA}{\end{eqnarray}}
\newcommand{\define}{\stackrel{\triangle}{=}}

\bibliographystyle{IEEEtran}
%\bibliographystyle{ieeetr}

\providecommand{\nCr}[2]{\,^{#1}C_{#2}} % nCr
\providecommand{\nPr}[2]{\,^{#1}P_{#2}} % nPr
\providecommand{\mbf}{\mathbf}
\providecommand{\pr}[1]{\ensuremath{\Pr\left(#1\right)}}
\providecommand{\qfunc}[1]{\ensuremath{Q\left(#1\right)}}
\providecommand{\sbrak}[1]{\ensuremath{{}\left[#1\right]}}
\providecommand{\lsbrak}[1]{\ensuremath{{}\left[#1\right.}}
\providecommand{\rsbrak}[1]{\ensuremath{{}\left.#1\right]}}
\providecommand{\brak}[1]{\ensuremath{\left(#1\right)}}
\providecommand{\lbrak}[1]{\ensuremath{\left(#1\right.}}
\providecommand{\rbrak}[1]{\ensuremath{\left.#1\right)}}
\providecommand{\cbrak}[1]{\ensuremath{\left\{#1\right\}}}
\providecommand{\lcbrak}[1]{\ensuremath{\left\{#1\right.}}
\providecommand{\rcbrak}[1]{\ensuremath{\left.#1\right\}}}
\theoremstyle{remark}
\newtheorem{rem}{Remark}
\newcommand{\sgn}{\mathop{\mathrm{sgn}}}
\providecommand{\abs}[1]{\left\vert#1\right\vert}
\providecommand{\res}[1]{\Res\displaylimits_{#1}} 
\providecommand{\norm}[1]{\lVert#1\rVert}
\providecommand{\mtx}[1]{\mathbf{#1}}
\providecommand{\mean}[1]{E\left[ #1 \right]}
\providecommand{\fourier}{\overset{\mathcal{F}}{ \rightleftharpoons}}
\providecommand{\ztrans}{\overset{\mathcal{Z}}{ \rightleftharpoons}}

%\providecommand{\hilbert}{\overset{\mathcal{H}}{ \rightleftharpoons}}
\providecommand{\system}{\overset{\mathcal{H}}{ \longleftrightarrow}}
	%\newcommand{\solution}[2]{\textbf{Solution:}{#1}}
\newcommand{\solution}{\noindent \textbf{Solution: }}
\providecommand{\dec}[2]{\ensuremath{\overset{#1}{\underset{#2}{\gtrless}}}}
\numberwithin{equation}{section}
%\numberwithin{equation}{subsection}
%\numberwithin{problem}{subsection}
%\numberwithin{definition}{subsection}
\makeatletter
\@addtoreset{figure}{problem}
\makeatother

\let\StandardTheFigure\thefigure
%\renewcommand{\thefigure}{\theproblem.\arabic{figure}}
\renewcommand{\thefigure}{\theproblem}



\def\putbox#1#2#3{\makebox[0in][l]{\makebox[#1][l]{}\raisebox{\baselineskip}[0in][0in]{\raisebox{#2}[0in][0in]{#3}}}}
     \def\rightbox#1{\makebox[0in][r]{#1}}
     \def\centbox#1{\makebox[0in]{#1}}
     \def\topbox#1{\raisebox{-\baselineskip}[0in][0in]{#1}}
     \def\midbox#1{\raisebox{-0.5\baselineskip}[0in][0in]{#1}}

\vspace{3cm}

\title{ 
	\logo{Voice Recognition through Machine Learing}
}
\author{Raktim Gautam Goswami$^{1}$, Abhishek Bairagi$^{2}$ \& G V V Sharma$^{3}$ 
%<-this  stops a space
\thanks{The authors are with the Department
of Electrical Engineering, Indian Institute of Technology, Hyderabad
502285 India .  e-mail: 1. ee17btech11004@iith.ac.in, 2. ee17btech11051@iith.ac.in,  
3. gadepall@iith.ac.in}% <-this % stops a space
%%\thanks{J. Doe and J. Doe are with Anonymous University.}% <-this % stops a space
%%\thanks{Manuscript received April 19, 2005; revised January 11, 2007.}}
}




% make the title area
\maketitle

%\newpage

\tableofcontents

\renewcommand{\thefigure}{\theenumi}
\renewcommand{\thetable}{\theenumi}


\bigskip

\begin{abstract}
%
This manual shows how to develop a voice recognition algorithm and use it to 
control a toycar. 
%
\end{abstract}


\section{Dataset}
%
\begin{enumerate}[label=\thesection.\arabic*
,ref=\thesection.\theenumi]

\item Record 'forward' 80 times using you phone and save as 'forwardi.wav' for $i 
= 1,\dots, 80$.
%
\item Repeat by recording 'left', 'right', 'back' and 'stop'. Make sure that the 
audio files for each command are in separate directories. Download the following 
directory for reference
\begin{lstlisting}
svn checkout https://github.com/gadepall/EE1390/trunk/AI-ML/audio_dataset
\end{lstlisting}
\item Suitably edit the following scirpt
\begin{lstlisting}
wget https://raw.githubusercontent.com/gadepall/EE1390/trunk/AI-ML/codes/
\end{lstlisting}
%
	%
%

\item Stick a 9V battery to the breadboard and connect the positive and negative 
terminals to extreme ends of the breadboard.
%
\item 
Stick a 9V battery to the breadboard and connect the positive and negative terminals to extreme ends of the breadboard.
%
\item
Provide 9V to the supply pin of the Arduino.
%
\item
Plug the L293D motor driver IC in Fig. \ref{fig:l293d} on the breadboard.

	%
%\begin{figure}[!ht]
%\begin{center}
%\includegraphics[width=\columnwidth]{./figs/l293d}
%\end{center}
%\caption{}
%\label{fig:l293d}
%\end{figure}
%
%
\item
Connect the L293D pins according to Table \ref{table:l293d}.

%	\input{./figs/l293d.tex}
\item
Connect the HC05 pins according to Table \ref{table:hc05}.


%\begin{figure}[ht!]
%\begin{center}
%\includegraphics[width=\columnwidth]{./figs/HC05}
%\end{center}
%\caption{HC05 Bluetooth module}
%\label{fig:hc05}
%\end{figure}
%
%	\input{./figs/hc05.tex}


\end{enumerate}
%
\section{Implementation}
\begin{enumerate}[label=\thesection.\arabic*
,ref=\thesection.\theenumi]
\item Dump the following code in Arduino using its IDE.
%
\item Install Google API "Arduino Bluetooth Controller" using google play-store
%\begin{figure}[!h]
%\begin{center}
%\includegraphics[width = \columnwidth]{./figs/app}
%%\includegraphics[width = 7cm, height = 10cm]{./figs/app}
%\end{center}
%\caption{}
%\label{fig:App}
%\end{figure}
\item Open the app and connect to HC-05.
\item Open voice control section in the app and tap to give following commands. 

\textit{Left, Right, Forward, Back \& Stop}

\end{enumerate}

\section{Building the neural network}
\subsection{Theory}
We have used linear regression in our model. Here, all the features are tried to be approximated using an n-dimensional straight line (n being the number of features). The equation used for this is 
\begin{equation}
\sum_{i} Wi*xi + b
\end{equation}
In matrix form it is $$ out = W.X + B $$ The output(out) is then put as input to the sigmoid function and the output of it is a number scaled between 0 and 1. This is the actual output(Y') we are interested in .  The sigmoid function is defined as $$sigmoid(x) = 1/(1+exp(-x))$$ The cost function is then calculated using mean squared error as $$ J = 0.5*(Y - Y')^2$$ Gradient descent algorithm is used to get minimum error using the derivative of the error(J) with respect to weight (W).This process is carried on for a number of times to get the best accuracy.
\paragraph{How is the descent algorithm obtained from the cost function?\newline}
We initialized the parameters W1 and b . Now we want Mean Square Error function to 
be minimum.The way we do this is by taking the derivative (the tangential line to a 
function) of our cost function with respect to each parameter. Derivative at that 
point and it will give us a direction to move towards. And then we update the value 
of all the parameters according to the derivative obtained.And then we iterate the 
process(number of itterations are decided by us ) .We make steps down the cost 
function in the direction with the steepest descent. The size of each step is 
determined by the parameter $\alpha$, which is called the learning rate.The gradient 
descent algorithm is repeated until convergence: 
$Mj :=Mj-(learningrate)*(delta Loss)*input$
%$Mj :​= Mj​ - (learningrate)*(delta Loss)*input$  %%%%%%% to be updated


\subsection{Python code}

\url{https://github.com/raktimgg/ML-algorithm-for-speech-recognition/blob/master/code.py}\newline
This is the full code that is used for training. The accuracy we are getting is around 98 percent.

\subsection{Dataset}
We have made our own dataset by recording 25 samples of each word. Each of these samples are recreated by adding empty elements in the front and back in many different cobinations to create a dataset of 6250 samples for each word. All the audio files are imported to an array in the code and converted to mfcc format before training. For creating training dataset we recorded 25 audio file of each of the following word -\newline
1)Forward\newline
2)Left\newline
3)Right\newline
4)Back\newline
5)Stop\newline
The code for generating 6250 samples for each word from 25 samples can be found in the github link attached.\newline
\url{https://github.com/abhishekbairagi/Making-Dataset-for-ML/blob/master/250files.py}



\section{Transfering the weights to Raspberry Pi (Yet to be done)}
The weight(W1 and B) are saved in a file at the end of the code. These weights will be transferred to the raspberry pi and a simple program written, will record audio on the raspberry pi, do the calculations using the weights and predict the text output. This output will be sent,using bluetooth, to the toy car, which will move accordingly.

\end{document}


